\documentclass{article}
\usepackage[slovak]{babel}
\usepackage[utf8]{inputenc}

\usepackage{indentfirst}
\usepackage{graphicx}

\begin{document}
    \begin{titlepage}
        \begin{center}
            \textsc{\Huge Vysoké učení technické v~Brně\\
            		\huge Fakulta informačních technologií\\}
            \vspace{\stretch{0.382}}
            {\LARGE ISA - Síťové aplikace a~správa sítí\\}\vspace{2em}
            {\Large dokumentácia k 2.variante projektu\\}\vspace{2em}
            \Huge dns-export\\
            \vspace{\stretch{0.618}}
        \end{center}
        {\Large \today \hfill Tomáš Nereča}\vspace{-2em}
    \end{titlepage}

    \tableofcontents
        \thispagestyle{empty}
        \newpage
        \setcounter{page}{1}
    \newpage
    
    \section{Úvod}

        \textbf{dns-export} je aplikácia na spracovávanie dát protokolu \emph{DNS (Domain Name System)}.
        Aplikácia teda musí vedieť identifikovať a spracovať \emph{paket} daného protokolu.
        Jednotlivé pakety môže buď čerpať z \emph{pcap} súboru alebo odchytávať v reálnom čase
        zo sieťového rozhrania.

        Zo spracovaných dát vytvára štatistiky, ktoré v časovom intervale posiela protokolom \emph{Syslog}
        na centrálny logovací server alebo jednoducho vypisuje na štandardný výstup. 
    
    \section{Návrh aplikácie}

        \subsection{Objektový návrh}
        Svoju aplikáciu som sa rozhodol naprogramovať objektovo. Dodržiaval som štruktúru jedna \emph{trieda}
        na jeden zdrojový súbor. Každá trieda má svoju významom vyhradenú funkciu. 

        \subsection{Spracovanie paketov}
        Aplikácia spracuje vstupné argumenty a po ich validácii sa buď spustí spracovanie pcap súboru, alebo odchytávanie
        paketov zo sieťového rozhrania. Pre každý spracovávaný paket sa vytvorí objekt držiaci dáta paketu.

        V tomto objekte sa najprv spracujú jednotlivé hlavičky paketu. Ak sa zistí, že paket obsahuje odpovede na DNS otázky,
        Vytvorí sa pre každú odpoveď objekt, ktorý tieto odpovede spracováva.

        \subsection{Ukladanie štatistík}
        Po úspešnom spracovaní odpovedí sa každá odpoveď pridá do objektu, ktorý obsahuje kolekciu všetkých nazbieraných štatistík.
        Pomocou metódy sa zistí, či sa daná štatistická informácia v objekte už nachádza a v tom prípade sa len inkrementuje počítadlo.
        
        \subsection{Odosielanie a výpis štatistík}
        Po prečítaní z pcap súboru sa štatistiky uložené v objekte odošlú na server ak je zadaný alebo vypíšu na štandardný výstup.
        V prípade živého odchytávania sa pred jeho spustením vytvorí separátne vlákno, ktoré bude v zadanom časovom intervale
        odosielať všetky doposiaľ nazbierané štatistiky. Okrem toho bude aplikácia pri odchytení signálu \textbf{SIGUSR1} volať funkciu
        na výpis štatistík na štandardný výstup.
        
    \section{implementácia}
    
        \subsection{DnsExport}
        Po spustení aplikácie je vytvorený objekt tejto triedy a zavolaná funkcia \textbf{Main()}. Nastaví sa odchýtavanie signálov
        a na základne vstupných argumentov načítaných pomocou vstavanej funkcie \textsc{getopt()}\footnote{http://man7.org/linux/man-pages/man3/getopt.3.html} sa rozhodne, či sa spustí funkcia
        \textbf{sniffInterface()} alebo \textbf{sniffFile()}. 
        
        V prvom prípade sa pred započatím odchytávania spustí funkcia \textbf{sendingLoop()},
        ktorá sa v separátnom vlákne stará o pravidelné odosielanie na Syslog server. Pre odoslanie volá funkcia \textbf{sendStats()}, ktorá si volá
        pomocné funkcie na vytvorenie správ ako napr.: \textbf{getMessages()} alebo \textbf{getFormattedTime()} a následne \emph{UDP} protokolom 
        odosiela správy.

        V druhom prípade ak nebol zadaný Syslog server, je po prečítaní súboru zavolaná funkcia \textbf{getMessages()}, ktorá štatistiky vypíše
        na štandardný výstup a aplikácia končí. Rovnaká funkcia je volaná aj pri prijatí signálu \textbf{SIGUSR1}.

        O všetky akcie ohľadom odchytávania paketov sa stará knižnica \textbf{libpcap}\footnote{https://www.tcpdump.org/pcap.html}. Pre každý paket
        je volaná \emph{callback} funkcia \textbf{pcapHandler()}. Tam je vytvorený objekt triedy \textbf{DnsPacket} a zavolaná funkcia \textbf{Parse()}.

        Ak boli v pakete nájdené validné DNS odpovede, je zavolaná funkcia \textbf{addRecords()}, ktorá do globálnej premennej \textbf{recordList} pridá nový
        objekt triedy \textbf{DnsRecord}, ak sa daná štatistika v liste ešte nenachádza, alebo len zvýši počítadlo u nájdenej štatistiky.



    
    \section{Práca v~tíme}
    Ako tím sme sa začali schádzať prakticky ihneď po zaregistrovaní nášho zadania. Stretávali sme sa 
    zvyčajne raz do týždňa. Vždy sme prediskutovali aktuálny stav práve implementovaných častí 
    a ďalší postup či korekcie v~zdrojovom kóde. Ako komunikačný kanál sme využívali prevažne 
    skupinovú konverzáciu na sociálnej sieti Facebook. Bolo to jednoduché a efektívne riešenie.

    Prácu sme sa snažili rozdeľovať rovnomerne medzi všetkých členov tímu, čo sa nám nepodarilo vždy. 
    Po pridelení úlohy sme stanovili deadline, aby sme mohli čo najskôr pokračovať na nasledujúcej časti projektu. 
    Jednotlivé časti sme sa snažili implementovať paralelne s~preberanou látkou na prednáškach, aby sme sa vyhli chybám z dôvodu neznalosti.
    

        \subsection{Správa zdrojového kódu}
        Pre správu a zdielanie zdrojových súborov sme využili verzovací systém \emph{Git} a webovú službu GitHub pre vzdialené ukladanie, ktorú sme už počas štúdia využili na verzovanie projektov v~iných predmetoch.
        
        Na sledovanie postupu pri plnení pridelených úloh a oznamovanie nájdených chýb 
        sme využívali nástroj Trello, ktorý slúži ako online kanban pre sledovanie projektov. 
        
        Vďaka týmto informačným kanálom mohli mať všetci členovia tímu prístup k~najaktuálnejšej
        verzii projektu a reagovať na vzniknuté chyby efektívne.

        \subsection{Rozdelenie práce v tíme}

        \textbf{Tomáš Nereča} pracoval na implementácii lexikálneho analyzátora a v neskoršej fáze vývoja pomáhal pri implementácii niektorých
        funkcií precedenčnej syntaktickej analýzy. Okrem toho sa podieľal na menších moduloch symtable, errors, stack a ifunc.  

        \textbf{Jiří Vozár} mal na starosti vedenie implementácie, návrh komunikácie medzi modulmi a implementáciu syntaktickej analýzy rekurzívnym zostupom.

        \textbf{Samuel Obuch} pracoval na implementácii precedenčnej syntaktickej analýzy, vytvoril regresné testy a vypracoval pomocný modul strings pre lexikálny analyzátor.

        \textbf{Ján Farský} vypracoval základné jednotkové testy pre lexikálny analyzátor, vytvoril precedenčnú tabuľku, pracoval na module ifunc a napísal základ dokumentácie.

        Dôvodom nerovnomerného rozdelenia bodov bolo nedostatočné splnenie úloh jedného 
        z~členov tímu, čo viedlo k nutnosti zapojiť k týmto úlohám iných členov tímu.

    \section{Záver}
    S~projektom podobného rozsahu sa ešte nikto z~nás predtým nestretol, preto ho považujeme za dobrú
    skúsenosť pre každého z~nás. Pri jeho riešení sme prakticky využili získané vedomosti z~predmetov 
    IFJ a IAL.
    
    Pre správne fungovanie tímu bolo potrebné kvalitné riadenie a pridelovanie úloh a pravideľná
    komunikácia medzi jednotlivými členmi tímu. Výsledkom tejto práce je funkčný a z~nášho pohľadu vydarený 
    prekladač jazyka IFJ17.
    
    \newpage
    \section{Prílohy}
        \subsection{Diagram konečného automatu}
            \newpage
            
        \subsection{LL gramatika}
            \begin{enumerate}
                % NESAHAT - zhenodnotí čísla v tabulce LL gramatiky!
                \item \texttt{<program> -> declare function <functionDecl> eol <program>}
                \item \texttt{<program> -> function <functionDef> eol <program>}
                \item \texttt{<program> -> scope <statementList> end scope}
                
                \item \texttt{<functionDecl> -> <functionHeader>}
                \item \texttt{<functionDef> -> <functionHeader> eol <statementList> end function}
                
                \item \texttt{<functionHeader> -> identifier ( <functionParams> ) as type}
                
                \item \texttt{<functionParams> -> <functionParam> <nextFuncParam>}
                \item \texttt{<functionParams> -> epsilon}
                
                \item \texttt{<nextFuncParam> -> , <functionParam> <nextFuncParam>}
                \item \texttt{<nextFuncParam> -> epsilon}
                
                \item \texttt{<functionParam> -> identifier as type}
                
                \item \texttt{<statementList> -> <statement> eol <statementList>}
                \item \texttt{<statementList> -> epsilon}
                
                \item \texttt{<statement> -> dim <declaration>}
                \item \texttt{<statement> -> identifier = <expression>}
                \item \texttt{<statement> -> input identifier}
                \item \texttt{<statement> -> print <printArgs>}
                \item \texttt{<statement> -> if <expression> then eol <statementList> <else>}
                \item \texttt{<statement> -> do while <expression> eol <statementList> loop}
                \item \texttt{<statement> -> return identifier}
                
                \item \texttt{<declaration> -> identifier as type}
                \item \texttt{<declaration> -> identifier as type = <expression>}
                
                \item \texttt{<printArgs> -> <expression> ;}
                \item \texttt{<printArgs> -> <expression> ; <printArgs>}
                
                \item \texttt{<else> -> elseif <expression> then eol <else> end if}
                \item \texttt{<else> -> else eol <statementList> end if}
                \item \texttt{<else> -> end if}
                
                \item \texttt{<expression> ->} vyhodnocuje se precedenční syntaktickou analýzou
            \end{enumerate}
        \newpage
        
        \subsection{LL tabuľka}
        \newcommand{\tterm}[1]{\rotatebox[origin=c]{90}{\texttt{#1}}}
            \begin{tabular}{|r|*{10}{c|}}
                \hline
                & \tterm{declare} & \tterm{function} & \tterm{scope} & \tterm{identifier} & \tterm{dim} &
                \tterm{input} & \tterm{print} & \tterm{if} & \tterm{do} & \tterm{return} \\\hline \hline
                \texttt{<program>} & 1 & 2 & 3 &&&&&&& \\\hline
                \texttt{<functionDecl>} &&&& 4 &&&&&& \\\hline
                \texttt{<functionDef>} &&&& 5 &&&&&& \\\hline
                \texttt{<functionHeader>} &&&& 6 &&&&&& \\\hline
                \texttt{<functionParams>} &&&& 7, 8 &&&&&& \\\hline
                \texttt{<nextFuncParam>} &&&&&&&&&& \\\hline
                \texttt{<functionParam>} &&&& 11 &&&&&& \\\hline
                \texttt{<statementList>} &&&& 12 & 12 & 12 & 12 & 12 & 12 & 12 \\\hline
                \texttt{<statement>} &&&& 15 & 14 & 16 & 17 & 18 & 19 & 20 \\\hline
                \texttt{<declaration>} &&&& 21, 22&&&&&& \\\hline
                \texttt{<else>} &&&& 23, 24&&&&&& \\\hline
            \end{tabular}
            
            \begin{tabular}{|r|*{9}{c|}}
                \hline
                & \tterm{elseif} & \tterm{else} & \tterm{end} & \tterm{loop} & \tterm{eol} &
                \tterm{(} & \tterm{)} & \tterm{=} & \tterm{,} \\\hline \hline
                \texttt{<program>} &&&&&&&&& \\\hline
                \texttt{<functionDecl>} &&&&&&&&& \\\hline
                \texttt{<functionDef>} &&&&&&&&& \\\hline
                \texttt{<functionHeader>} &&&&&&&&& \\\hline
                \texttt{<functionParams>} &&&&&&& 8 && \\\hline
                \texttt{<nextFuncParam>} &&&&&&& 10 && 9 \\\hline
                \texttt{<statementList>} & 13 & 13 & 13 & 13 &&&&& \\\hline
                \texttt{<statement>} &&&&&&&&& \\\hline
                \texttt{<declaration>} &&&&&& 23, 24&&& \\\hline
                \texttt{<else>} & 25 & 26 & 27 &&&&&& \\\hline
            \end{tabular}
        \newpage

        \subsection{Precedenčná tabuľka}

        \begin{center}
        \begin{tabular}{|c||c|c|c|c|c|c|c|c|c|c|c|c|c|c|c|c|c|c|}
        \hline
                    &  =  & $<>$ & $<$= & $>$= & $<$ & $>$ &  +  &  -  &  *   &  /  & \textbackslash &  (  &  )  & \$  \\ 
        \hline
        \hline  
           =        & $>$ &  $>$ &  $>$ &  $>$ & $>$ & $>$ & $<$ & $<$ &  $<$ & $<$ & $<$            & $<$ & $>$ & $>$ \\ 
        \hline  
          $<>$      & $>$ &  $>$ &  $>$ &  $>$ & $>$ & $>$ & $<$ & $<$ &  $<$ & $<$ & $<$            & $<$ & $>$ & $>$ \\
        \hline  
          $<$=      & $>$ &  $>$ &  $>$ &  $>$ & $>$ & $>$ & $<$ & $<$ &  $<$ & $<$ & $<$            & $<$ & $>$ & $>$ \\
        \hline  
          $>$=      & $>$ &  $>$ &  $>$ &  $>$ & $>$ & $>$ & $<$ & $<$ &  $<$ & $<$ & $<$            & $<$ & $>$ & $>$ \\
        \hline  
          $<$       & $>$ &  $>$ &  $>$ &  $>$ & $>$ & $>$ & $<$ & $<$ &  $<$ & $<$ & $<$            & $<$ & $>$ & $>$ \\
        \hline        
          $>$       & $>$ &  $>$ &  $>$ &  $>$ & $>$ & $>$ & $<$ & $<$ &  $<$ & $<$ & $<$            & $<$ & $>$ & $>$ \\
        \hline  
           +        & $>$ &  $>$ &  $>$ &  $>$ & $>$ & $>$ & $>$ & $>$ &  $<$ & $<$ & $<$            & $<$ & $>$ & $>$ \\
        \hline  
           -        & $>$ &  $>$ &  $>$ &  $>$ & $>$ & $>$ & $>$ & $>$ &  $<$ & $<$ & $<$            & $<$ & $>$ & $>$ \\ 
        \hline  
          */        & $>$ &  $>$ &  $>$ &  $>$ & $>$ & $>$ & $>$ & $>$ &  $>$ & $>$ & $>$            & $<$ & $>$ & $>$ \\ 
        \hline  
\textbackslash      & $>$ &  $>$ &  $>$ &  $>$ & $>$ & $>$ & $>$ & $>$ &  $<$ & $<$ & $>$            & $<$ & $>$ & $>$ \\
        \hline  
           (        & $<$ &  $<$ &  $<$ &  $<$ & $<$ & $<$ & $<$ & $<$ &  $<$ & $<$ & $<$            & $<$ &  =  &     \\  
        \hline  
           (        & $>$ &  $>$ &  $>$ &  $>$ & $>$ & $>$ & $>$ & $>$ &  $>$ & $>$ & $>$            &     & $>$ & $>$ \\ 
        \hline  
          \$        & $<$ &  $<$ &  $<$ &  $<$ & $<$ & $<$ & $<$ & $<$ &  $<$ & $<$ &  $<$            & $<$ &     &     \\ 
        \hline  
        \end{tabular}
    \end{center}
\end{document}