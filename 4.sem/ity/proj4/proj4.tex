\documentclass[a4paper, 11pt]{article}
\usepackage[czech]{babel}
\usepackage[utf8]{inputenc}
\usepackage{times}
\usepackage{multirow}
\usepackage[left=2cm,text={17cm, 24cm},top=3cm]{geometry}
\bibliographystyle{czplain}

\begin{document}
\begin{titlepage}
\begin{center}
\textsc{\Huge Vysoké učení technické v~Brně\\
		\huge Fakulta informačních technologií\\}
\vspace{\stretch{0.382}}
{\LARGE Typografie a~publikování\,--\,4. projekt\\}
\Huge Fonty\\
\vspace{\stretch{0.618}}
\end{center}
{\Large \today \hfill Tomáš Nereča}\vspace{-2em}
\end{titlepage}

\section{Úvod}
Fontom sa nazýva grafická podoba textu, t.j. typ písma, jeho veľkosť, šírka, farba alebo celkový dizajn. \cite{computerhope:font}

\section{História}
Vývoj typografie a tým pádom aj fontu začal asi v treťom storočí nášho letopočtu. Vtedy sme poznali jeden typ písma a to tzv. rímske kapitálky. 
V roku 2011 bolo známych asi 200 000 rôznych fontov, pričom väčšina z nich vychádza z písma \emph{Garamond}. \cite{felici:typography}

Toto písmo pochádza od Parížana menom Claude Garamond, ktorý bol prvým nezávislým tvorcom písma. Garamond si na tvorbe písma dokonca založil živnosť. Dokázal tak oddeliť samotnú tvorbu písma od tlače. \cite{uhlirova:typografie}

Významnou udalosťou vo vývoji fontu je vynález kníhtlače. Pôvodne sa tlačiari snažili napodobňovať písané písmo, čo sa neskôr obrátilo a práve tlačené písmo sa stalo akousi predlohou pre písané písmo. \cite{wikisofia:vyvoj_pisma}

\section{Delenie}

\section{Konkrétne fonty}
Font \textbf{Times New Roman} navrhol v roku 1931 Victor Lardent. Tento font patrí k rodine \emph{Serif}, ide teda o pätkové písmo. Tento font mal veľký vplyv na vznik ďaľších fontov aj po nástupe digitálnych médií. \cite{typedia:times}

Font \textbf{Helvetica}, vytvorený v roku 1957 patrí naopak k rodine bezpätkových písem \emph{Sans-serif}. Má neutrálny vzhľad a často sa využíva na komerčné účely. \cite{magazine:font}

\textbf{Courier} z roku 1955 je neproporcionálne font typu \emph{Egyptienka}. V 90-tych rokoch minulého storočia sa začal používať v elektronike, kedy bolo treba zaistiť zarovnanie stĺpcov. \cite{garfield:my_type} To bolo zaistené tým, že je font neproporcionálny, teda každé písmeno zaberá na riadku rovnakú šírku.


\section{Zaujímavosti}
Aký font vybrať pre výučbové účely? Výskumy zistili, že človek si zapamätá viac informácií z textu, ktorý je napísaný nepríjemným, ťažko čitatelným fontom. \cite{ny:come_on}







\bibliography{proj4}
\end{document}