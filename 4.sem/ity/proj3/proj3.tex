\documentclass[a4paper, 11pt]{article}
\usepackage[czech]{babel}
\usepackage[utf8]{inputenc}
\usepackage[IL2]{fontenc}
\usepackage{times}
\usepackage{multirow}
\usepackage[left=2cm,text={17cm, 24cm},top=3cm]{geometry}

\begin{document}
	\begin{titlepage}
		\begin{center}
			
			\textsc{\Huge Vysoké učení technické v~Brně\\[0.4em]}
				\huge Fakulta informačních technologií\\
			\vspace{\stretch{0.378}}
			{\LARGE Typografie a~publikování\,--\,3. projekt\\[0.3em]}
				\Huge Tabulky a~obrázky\\
			\vspace{\stretch{0.622}}
		\end{center}
		{\Large\today \hfill Tomáš Nereča}
	\end{titlepage}

\section{Úvodní strana}
Název práce umístěte do zlatého řezu a nezapomeňte uvést dnešní datum a vaše jméno a příjmení.

\section{Tabulky}
Pro sázení tabulek můžeme použít buď prostředí \texttt{tabbing} nebo prostředí \texttt{tabular}.

\subsection{Prostředí \texttt{tabbing}}
Při použití \texttt{tabbing} vypadá tabulka následovně:

\begin{tabbing}
	Vodní melouny \quad \= \textbf{Cena} \quad \= \textbf{Množství} \kill
	\textbf{Ovoce} \> \textbf{Cena} \> \textbf{Množství}\\
	Jablka \> 25,90 \> 3\ kg\\
	Hrušky \> 27,40 \> 2,5\ kg\\
	Vodní melouny \> 35,-- \> 1\ kus
\end{tabbing}

Toto prostředí se dá také použít pro sázení algoritmů, ovšem vhodnější je použít prostředí \texttt{algorithm} nebo
\texttt{algorithm2e} (viz sekce 3).

\subsection{Prostředí \texttt{tabular}}
Další možností, jak vytvořit tabulku, je použít prostředí \texttt{tabular}. Tabulky pak budou vypadat takto\footnote{Kdyby byl problem s \texttt{cline}, zkuste se podívat třeba sem: http://www.abclinuxu.cz/tex/poradna/show/325037.}:

\begin{table}[h]
\begin{center}
\catcode`\-=12 % http://www.abclinuxu.cz/tex/poradna/show/325037
\begin{tabular}{|c|c|c|} \hline 
	& \multicolumn{2}{|c|}{\textbf{Cena}}\\ \cline{2-3}
	\textbf{Měna} & \textbf{Nákup} & \textbf{Prodej}\\ \hline
EUR & 27,02 & 27,20 \\
GBP & 31,08 & 31,80 \\
USD & 25,15 & 25,51 \\ \hline
\end{tabular}
\caption{Tabulka kurzů k~dnešnímu dni}
\label{kurzy}
\end{center}
\end{table}

\begin{table}[h]
\begin{center}
\catcode`\-=12
\begin{tabular}{|c|c|} \hline
$A$ & $\neg A$ \\ \hline
\textbf{P} & N \\ \hline
\textbf{O} & O \\ \hline
\textbf{X} & X \\ \hline
\textbf{N} & P \\ \hline
\end{tabular}
\begin{tabular}{|c|c|c|c|c|c|} \hline
			\multicolumn{2}{| c |}{\multirow{2}{*}{$A \wedge B$}} & \multicolumn{4}{| c |}{$B$}\\ \cline{3-6}
			\multicolumn{2}{| c |}{} & \textbf{P} & \textbf{O} & \textbf{X} & \textbf{N}\\ \hline
			\multirow{4}{*}{$A$} &\textbf{P} & P & O~& X & N\\ \cline{2-6}
			&\textbf{O} & O~& O~& N & N\\ \cline{2-6}
			&\textbf{X} & X & N & X & N\\ \cline{2-6}
			&\textbf{N} & N & N & N & N\\ \hline
		\end{tabular}
		%%
		\begin{tabular}{| c | c | c | c | c | c |} \hline
			\multicolumn{2}{| c |}{\multirow{2}{*}{$A \vee B$}} & \multicolumn{4}{| c |}{$B$}\\ 
			\cline{3-6}
			\multicolumn{2}{| c |}{} & \textbf{P} & \textbf{O} & \textbf{X} & \textbf{N}\\ \hline
			\multirow{4}{*}{$A$} &\textbf{P} & P & P & P & P\\ \cline{2-6}
			&\textbf{O} & P & O~& P & O\\ \cline{2-6}
			&\textbf{X} & P & P & X & X\\ \cline{2-6}
			&\textbf{N} & P & O~& X & N\\ \hline
		\end{tabular}
		%%
		\begin{tabular}{| c | c | c | c | c | c |} \hline
			\multicolumn{2}{| c |}{\multirow{2}{*}{$A \rightarrow B$}} & \multicolumn{4}{| c |}{$B$}\\ \cline{3-6}
			\multicolumn{2}{| c |}{} & \textbf{P} & \textbf{O} & \textbf{X} & \textbf{N}\\ \hline
			\multirow{4}{*}{$A$} &\textbf{P} & P & O~& X & N\\ \cline{2-6}
			&\textbf{O} & P & O~& P & O\\ \cline{2-6}
			&\textbf{X} & P & P & X & X\\ \cline{2-6}
			&\textbf{N} & P & P & P & P\\ \hline
		\end{tabular}
		
		\caption{Protože Kleeneho trojhodnotová logika už je \uv{zastaralá}, 
			uvádíme si zde příklad čtyřhodnotové logiky}
		\label{logika}
	\end{center}
\end{table}
\end{document}