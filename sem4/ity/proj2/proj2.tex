\documentclass[a4paper, 11pt, twocolumn]{article}
\usepackage[czech]{babel}
\usepackage[utf8]{inputenc}
\usepackage[IL2]{fontenc}
\usepackage{times}
\usepackage[left=1.5cm,text={18cm, 25cm},top=2.5cm]{geometry}
\newcommand{\myuv}[1]{\quotedblbase #1\textquotedblleft}

\begin{document}
\begin{titlepage}
\begin{center}
	
	\textsc{\Huge Fakulta informačních technologií\\[3.5mm]
			Vysoké učení technické v~Brně}
	\\[79mm]
	{\LARGE Typografie a publikování\,--\,2. projekt\\[1.5mm]
	Sazba dokumentů a matematických výrazů}
	\vfill
\end{center}
{\Large 2018 \hfill Tomáš Nereča(xnerec00)}
\\[-4mm]
\end{titlepage}

\section*{Úvod}
V~této úloze si vyzkoušíme sazbu titulní strany, matematických vzorců, prostředí a dalších textových struktur obvyklých pro technicky zaměřené texty (například rovnice (1) nebo definice 1 na straně 1). Rovněž si vyzkoušíme používání odkazů \verb|\ref| a \verb|\pageref|.

Na titulní straně je využito sázení nadpisu podle optického středu s~využitím zlatého řezu. Tento postup byl probírán na přednášce. Dále je použito odřádkování se
zadanou relativní velikostí 0.4em a 0.3em.

\section{Matematický text}
Nejprve se podíváme na sázení matematických symbolů a výrazů v plynulém textu včetně sazby definic a vět s využitím balíku \verb|amsthm|. Rovněž použijeme poznámku pod čarou s použitím příkazu \verb|\footnote| . Někdy je vhodné použít konstrukci \verb|${}$|, která říká, že matematický text nemá být zalomen.

\end{document}