\documentclass[a4paper, 11pt, twocolumn]{article}
\usepackage[czech]{babel}
\usepackage[utf8]{inputenc}
\usepackage[IL2]{fontenc}
\usepackage{times}
\usepackage{amsmath, amsthm, amsfonts}
\usepackage[left=1.5cm,text={18cm, 25cm},top=2.5cm]{geometry}

\theoremstyle{plain}
\newtheorem{definition}{Definice}
\theoremstyle{plain}
\newtheorem{sentence}{Věta}

\begin{document}
\begin{titlepage}
\begin{center}
	
	\textsc{\Huge Fakulta informačních technologií\\[3.5mm]
			Vysoké učení technické v~Brně}
	\\[79mm]
	{\LARGE Typografie a~publikování\,--\,2. projekt\\[2mm]
	Sazba dokumentů a~matematických výrazů}
	\vfill
\end{center}
{\Large 2018 \hfill Tomáš Nereča(xnerec00)}
\end{titlepage}

\section*{Úvod}
V~této úloze si vyzkoušíme sazbu titulní strany, matematických vzorců, prostředí a~dalších textových struktur obvyklých pro technicky zaměřené texty (například rovnice~\eqref{rovnice_1} nebo Definice~\ref{definice_1} na straně~\pageref{definice_1}). Rovněž si vyzkoušíme používání odkazů \verb|\ref| a~\verb|\pageref|.

Na titulní straně je využito sázení nadpisu podle optického středu s~využitím zlatého řezu. Tento postup byl probírán na přednášce. Dále je použito odřádkování se
zadanou relativní velikostí 0.4em a~0.3em.

\section{Matematický text}
Nejprve se podíváme na sázení matematických symbolů a~výrazů v~plynulém textu včetně sazby definic a~vět s~využitím balíku \verb|amsthm|. Rovněž použijeme poznámku pod čarou s~použitím příkazu \verb|\footnote|. Někdy je vhodné použít konstrukci \verb|${}$|, která říká, že matematický text nemá být zalomen.

\begin{definition} \label{definice_1}
\textnormal{Turingův stroj} (TS) je definován jako šestice tvaru $M=(Q,\Sigma,\Gamma,\delta, q_0, q_F)$, kde:
\begin{itemize}
\item $Q$ je konečná množina \textnormal{vnitřních (řídicích) stavů},
\item $\Sigma$ je konečná množina symbolů nazývaná \textnormal{vstupní abeceda}, $\Delta \not\in \Sigma$,
\item $\Gamma$ je konečná množina symbolů, $\Sigma\subset \Gamma$, $\Delta\in\Gamma$, nazývaná \textnormal{pásková abeceda},
\item $\delta:(Q\setminus\{q_F\})\times\Gamma\rightarrow Q\times(\Gamma\cup\{L,R\})$, kde\ $L,R\not\in\Gamma$, je parciální \textnormal{přechodová funkce},
\item $q_0$ je \textnormal{počáteční stav}, $q_0\in Q$ a
\item $q_F$ je \textnormal{koncový stav}, $q_F\in Q$.
\end{itemize}
\end{definition}

Symbol $\Delta$ značí tzv. \emph{blank} (prázdný symbol), který se vyskytuje na místech pásky, která nebyla ještě použita (může ale být na pásku zapsán i později).

\emph{Konfigurace pásky} se skládá z~nekonečného řetězce,
který reprezentuje obsah pásky a~pozice hlavy na tomto řetězci. Jedná se o~prvek množiny $\{\gamma \Delta^\omega\mid\gamma\in\Gamma^*\}\times\mathbb{N}$.\footnote{Pro libovolnou abecedu $\Sigma$ je $\Sigma^\omega$ množina všech \emph{nekonečných} řetězců nad $\Sigma$, tj. nekonečných posloupností symbolů ze $\Sigma$. Pro připomenutí: $\Sigma^*$ je množina všech \emph{konečných} řetězců nad $\Sigma$.}
\emph{Konfiguraci pásky} obvykle zapisujeme jako $\Delta xyz\underline{z}x\Delta...$ (podtržení značí pozici hlavy). \emph{Konfigurace stroje} je pak dána stavem řízení a~konfigurací pásky. Formálně se jedná o~prvek množiny $Q\times\{\gamma\Delta^\omega\mid\gamma\in\Gamma^*\}\times\mathbb{N}$.

\subsection{Podsekce obsahující větu a~odkaz}
\begin{definition} \label{definice_2}
\textnormal{Řetězec $w$ nad abecedou $\Sigma$ je přijat TS $M$} jestliže $M$ při aktivaci z~počáteční konfigurace pásky $\underline{\Delta}w\Delta...$ a~počátečního stavu $q_0$ zastaví přechodem do koncového stavu $q_F$, tj. $( q_0,\Delta w\Delta^\omega,0) \overset{*}{\underset{M}{\vdash}} (q_F,\gamma,n)$ pro nějaké $\gamma \in \Gamma^*$ a~$n\in\mathbb{N}$.

Množinu $L(M)=\{w\mid w\ \text{je přijat TS} \ M\}\subseteq\Sigma^*$ nazýváme \textnormal{jazyk přijímaný TS $M.$}
\end{definition}
Nyní si vyzkoušíme sazbu vět a~důkazů opět s~použitím
balíku \verb|amsthm|.
\begin{sentence} \label{veta_1}
Třída jazyků, které jsou přijímány TS, odpovídá \textnormal{rekurzivně vyčíslitelným jazykům}.
\end{sentence}
\begin{proof} \label{dukaz_1}
V~důkaze vyjdeme z~Definice \ref{definice_1} a~\ref{definice_2}.
\end{proof}

\section{Rovnice a~odkazy}
Složitější matematické formulace sázíme mimo plynulý text. Lze umístit několik výrazů na jeden řádek, ale pak je třeba tyto vhodně oddělit, například příkazem \verb|\quad|.

$$
\sqrt[i]{x_i^3}\quad\text{kde}\ x_i\ \text{je}\ i\text{-té sudé číslo} \quad y_i^{2\cdot y_i}\neq y_i^{y_i^{y_i}}
$$

V~rovnici \eqref{rovnice_1} jsou využity tři typy závorek s~různou explicitně definovanou velikostí.

\begin{align} \label{rovnice_1}
&x\quad=\quad\bigg\{\Big(\big[a + b\big] * c\Big)^d \oplus 1 \,\bigg\}\\
\nonumber &y\quad=\quad\lim_{x\to\infty}\frac{\sin^2x+ \cos^2x}{\frac{1}{\log_{10}{x}}}
\end{align}

V~této větě vidíme, jak vypadá implicitní vysázení limity $\lim_{n\to\infty}f(n)$ v~normálním odstavci textu. Podobně je to i s~dalšími symboly jako $\sum_{i=1}^{n} 2^{i}$ či $\bigcup_{A\in \mathcal{B}} A$. V~případě vzorců $\lim\limits_{n\to\infty} f(n)$ a~$\sum\limits_{i=1}^{n} 2^{i}$ jsme si vynutili méně úspornou sazbu příkazem \verb|\limits|.


\begin{align}\label{rovnice_2}
\int\displaylimits_{a}^{b}f(x)\,\mathrm{d}x\quad&=\quad-\int_{b}^{a}g(x)\,\mathrm{d}x\\
\overline{\overline{A\lor B}}\quad&\Leftrightarrow\quad \overline{\overline{A}\land\overline{B}}
\end{align}

\section{Matice}
Pro sázení matic se velmi často používá prostředí \verb|array| a~závorky (\verb|\left|, \verb|\right|).

$$
\left(
\begin{array}{ccc}
a + b & \widehat{\xi + \omega} & \hat{\pi}\\
\vec{a}&\overleftrightarrow{AC}&\beta
\end{array}
\right)=1\!\iff\!\mathbb{Q}=\mathbb{R}
$$

$$
\mathbf{A}=\left\|
\begin{array}{cccc}
a_{11} & a_{12} & \dots & a_{1n}\\
a_{21} & a_{22} & \dots & a_{2n}\\
\vdots & \vdots &\ddots& \vdots\\
a_{m1} & a_{m2} & \dots & a_{mn}
\end{array}
\right\|=
\left|
\begin{array}{ll}
t&u\\
v&w
\end{array} 
\right|=tw-uv
$$

Prostředí \verb|array| lze úspěšně využít i jinde.

$$
\binom{n}{k} =
\left\{
\begin{array}{ll}
\frac{n!}{k!(n-k)!} &\ \text{pro}\ 0 \leq k\ \leq n\\
0 &\ \text{pro}\ k\ <0\ \text{nebo}\ k\ >n
\end{array}
\right.
$$

\section{Závěrem}
V případě, že budete potřebovat vyjádřit matematickou konstrukci nebo symbol a~nebude se Vám dařit jej nalézt v~samotném \LaTeX u, doporučuji prostudovat možnosti balíku maker \AmS-\LaTeX.

\end{document}