\documentclass[a4paper, 11pt, twocolumn]{article}
\usepackage[czech]{babel}
\usepackage[utf8]{inputenc}
\usepackage[IL2]{fontenc}
\usepackage{times}
\usepackage{amsmath}
\usepackage{amsthm}
\usepackage{amsfonts}
\usepackage[left=1.5cm,text={18cm, 25cm},top=2.5cm]{geometry}
\newcommand{\myuv}[1]{\quotedblbase #1\textquotedblleft}

\theoremstyle{definition}
\newtheorem{definition}{Definice}

\begin{document}
\begin{titlepage}
\begin{center}
	
	\textsc{\Huge Fakulta informačních technologií\\[3.5mm]
			Vysoké učení technické v~Brně}
	\\[79mm]
	{\LARGE Typografie a publikování\,--\,2. projekt\\[1.5mm]
	Sazba dokumentů a matematických výrazů}
	\vfill
\end{center}
{\Large 2018 \hfill Tomáš Nereča(xnerec00)}
\\[-4mm]
\end{titlepage}

\section*{Úvod}
V~této úloze si vyzkoušíme sazbu titulní strany, matematických vzorců, prostředí a dalších textových struktur obvyklých pro technicky zaměřené texty (například rovnice (1) nebo definice 1 na straně 1). Rovněž si vyzkoušíme používání odkazů \verb|\ref| a \verb|\pageref|.

Na titulní straně je využito sázení nadpisu podle optického středu s~využitím zlatého řezu. Tento postup byl probírán na přednášce. Dále je použito odřádkování se
zadanou relativní velikostí 0.4em a 0.3em.

\section{Matematický text}
Nejprve se podíváme na sázení matematických symbolů a výrazů v plynulém textu včetně sazby definic a vět s využitím balíku \verb|amsthm|. Rovněž použijeme poznámku pod čarou s použitím příkazu \verb|\footnote| . Někdy je vhodné použít konstrukci \verb|${}$|, která říká, že matematický text nemá být zalomen.

\begin{definition}
\label{definice_1}
Turingův stroj \emph{(TS) je definován jako šestice tvaru $M=(Q,\Sigma,\Gamma,\delta, q_0, q_F)$, kde:}

\begin{itemize}
\item \emph{Q je konečná množina} vnitřních (řídicích) stavů,
\item \emph{$\Sigma$ je konečná množina symbolů nazývaná} vstupní abeceda, $\Delta \not\in \Sigma$,
\item \emph{ $\Gamma$ je konečná množina symbolů, $\Sigma \subset \Gamma, \Delta\in \Gamma$, nazývaná} pásková abeceda,
\item $\delta\!:\!( \,Q\setminus\{q_F\})\,\!\times\Gamma \longrightarrow Q\times( \,\Gamma\cup\{L,R\}) \,$,\emph{kde} $ L,R \not\in\Gamma,$ \emph{je parciální} přechodová funkce,
\item $q_0$ \emph{je} počáteční stav, $q_0\in Qa$
\item $q_F$ \emph{je} koncový stav, $q_F\in Qa$.
\end{itemize}

Symbol $\delta$ značí tzv. \emph{blank} (prázdný symbol), který se vyskytuje na místech pásky, která nebyla ještě použita (může ale být na pásku zapsán i později).

\emph{Konfigurace pásky} se skládá z nekonečného řetězce,
který reprezentuje obsah pásky a pozice hlavy na tomto řetězci. Jedná se o prvek množiny $\{\gamma \Delta^\omega\mid\gamma\in\Gamma^*\}\times \mathbb{N}$
\footnote{Pro libovolnou abecedu $\Sigma$ je $\Sigma^\omega$ množina všech nekonečných řetězců nad $\Sigma$, tj. nekonečných posloupností symbolů ze $\Sigma$. Pro připomenutí: $\Sigma^*$ je množina všech \emph{konečných} řetězců nad $\Sigma$.}.
\emph{Konfiguraci pásky} obvykle zapisujeme jako $\Delta xyz\underline{x}x\Delta\,\dots$ (podtržení značí pozici hlavy). \emph{Konfigurace stroje} je pak dána stavem řízení a konfigurací pásky. Formálně se jedná o prvek množiny $Q\times\{\gamma\Delta^\omega\mid\gamma\in\Gamma^*\}\times\mathbb{N}$.


\end{definition}

\end{document}